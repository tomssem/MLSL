
	\documentclass[xcolor=table]{beamer}
	\usepackage[utf8]{inputenc}
	\usepackage{default}
\usepackage{subcaption}
\usepackage{tikz}
\usetikzlibrary{spy,calc,positioning}
\usepackage{graphicx}
	\usepackage{pgfmath,pgffor}
%	\usepackage{enumitem}
	
	\usepackage{amsmath}
	\usepackage{amssymb}
	\usepackage{subcaption}
	\usepackage{mwe}
	\usepackage{soul}
	\usepackage[T1]{fontenc}
	\usepackage{lmodern}
	
	\usepackage{multicol}
	\usepackage{cleveref}
	
	% allow lowercase mathcal
	\usepackage[cal=boondoxo]{mathalfa}
%Decreases space before equations
%\usepackage{nccmath}
	
	
	
	% reset itemize style to Beamer style, enumitem overriddes it
%	\setitemize{label=\usebeamerfont*{itemize item}%
%		\usebeamercolor[fg]{itemize item}
%		\usebeamertemplate{itemize item}}
	
	\usepackage{utopia} %font utopia imported
	
	\usetheme{Madrid}
	\usecolortheme{default}
	
	\newcommand*{\FigFolder}{../figs}%
	
	\title[Coordinated Autonomous Vehicle Safety]{Multi-lane safety logic}
	\author{Tom Nicholson}
	\institute{Seoul National University}
	
	\usepackage{xcolor,pifont}
	\newcommand*\colourcheck[1]{%
		\expandafter\newcommand\csname #1check\endcsname{\textcolor{#1}{\ding{52}}}%
	}
	\newcommand*\colourqmark[1]{%
		\expandafter\newcommand\csname #1qmark\endcsname{\textcolor{#1}{\textbf{?}}}%
	}
	\colourqmark{red}
	\colourcheck{blue}
	\colourcheck{green}
	\colourcheck{red}
	
	\begin{document}
		
	\frame{\titlepage}
	
\begin{frame}{Safety for vehicles}
\begin{block}{High-level}
	Guarantee that vehicles have do not collide by maintaining safe stopping distance between them.
\end{block}
\pause

\begin{block}{In MLSL:}
	Different vehicles do not reserve the same space:
	\pause
	\begin{equation*}
	\mathcal{TS} \models \forall c, d: c \ne d \Rightarrow \neg \langle re(c) \land re(d)\rangle
	\end{equation*}
	For a given traffic snapshot, for all different vehicles it is not true that somewhere they reserve the same space
\end{block}

\end{frame}



\begin{frame}{Interval temporal logic}
Same as for temporal logic, except:

\begin{itemize}
	\item \textit{interval}, a non-empty finite sequence of states $s_{0..n}$
	\pause
	\item \textit{chop} operator: $s_{0..n} \models \varphi \frown \psi$
	\begin{block}{The \textit{chop} operator}
		Sequence can be ``chopped'' so the first formula holds over the prefix and the second over the suffix
		\begin{gather*}
		s_{0..n} \models \varphi \frown \psi \text{  iff  } \exists i \le n \text{   such that    } s_{0..i} \models \varphi \text{ and } s_{i..n} \models \psi
		\end{gather*}
		
	\end{block}
	\pause
	\item other interesting stuff not relevant to us
\end{itemize}

\end{frame}  

\begin{frame}{MLSL extends ITL (and ignores some of it)}

\begin{block}{Lane}
	Represented as an infinite \textit{interval} over $\mathbb{R}$
\end{block}
\pause
\begin{block}{Multi-lane road}
	A finite set of lanes
\end{block}
\pause
\begin{block}{Position of car}
	A point along the lanes
\end{block}


\end{frame}

\begin{frame}{Show basic motorway}
Show how MLSL can represent infinite view of a 3 lane motorway

Just vehicles with their position, speed and acceleration.

Is interval infinite?
\end{frame}

\begin{frame}{Views}
Snapshot of motorway

Can have a view with an ego vehicle

Only need to consider vehicles that are in this view, or intend to change into this view

Can change \cref{eq:safety_goal} to be defined over ego-vehicle views:
\begin{equation}\label{eq:safety_goal}
\mathcal{TS}, V_{ego} \models \neg \exists c \neq ego \land \langle re(ego) \land re(c)\rangle
\end{equation}
\end{frame}

\begin{frame}{Reservations}
Portion of road a vehicle needs to drive safely

May need two when changing lanes

Define standard view of a vehicle
\end{frame}

\begin{frame}{Claims}
Vehicles registering desire to change to a given lane
\end{frame}

\begin{frame}{Traffic Snapshot}
Put it all together into a state
\end{frame}

\begin{frame}{Transitions}
Transitions that are available
\end{frame}

\begin{frame}{Controllers}
Created by using transition guards and state invariants.
\end{frame}

\begin{frame}{Proofs}
By induction on number of transitions to reach $\mathcal{TS}$

Assumptions

Boils down to proving
\end{frame}

\end{document}

